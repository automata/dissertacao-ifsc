\begin{resumo2}
\vspace{-10mm}
VIEIRA, V. \textbf{\ABNTtitulodata}. 2014. 163p. Dissertação (Mestrado) - Instituto de Física de São Carlos, Universidade de São Paulo, São Carlos, 2014.
\vspace{15mm}

Enquanto muitos estudos são feitos para a análise e classificação de
pinturas e outros ramos das Artes, este estudo não se detém somente à
classificação. Extende-se aqui um método de análise já aplicado à
Música e Filosofia~\cite{vieira}, representando a evolução da Pintura
como uma série temporal onde relações como \textit{oposição},
\textit{inovação} e \textit{dialética} são medidas
quantitativamente. Para isso, um \textit{corpus} de pinturas de 12
artistas reconhecidos do período Barroco e de movimentos da Arte
Moderna foram analisadas. Um conjunto de 99 atributos foi extraído e
os atributos que mais contribuíram para a classificação das pinturas
foram selecionados. O espaço de projeção obtido a partir desses
atributos criou a base para a análise de medidas. Essas medidas
quantitativas revelaram observações sobre a evolução dos estilos
artísticos, especialmente quando comparados com outras áreas do
conhecimento humano já analisados. A Música parece ter evoluído com a
influência de uma tradição mestre-aprendiz (i.e.\ por apresentar alta
dialética entre os músicos estudados). A Filosofia apresenta valores
altos de oposição entre os representantes escolhidos~\cite{vieira},
sugerindo que sua evolução apresenta constante oposição de ideias. Já
na Pintura nota-se um outro padrão: aumento constante da inovação,
baixa oposição entre membros do mesmo período artístico e picos de
oposição no momento de transição entre estes períodos. Uma diferença
entre os movimentos Barroco e movimentos da Arte Moderna foi também
percebido no espaço projetado: enquanto as pinturas barrocas aparecem
formando um agrupamento sobreposto, as pinturas modernas apresentam
quase nenhuma sobreposição e estão dispostas espalhadas ao longo da
projeção, mais que as barrocas. Essa observação encontra base na
história da Arte onde os pintores barrocos compartilham
características estéticas de suas pinturas, enquanto os modernos
tendem a desenvolver seus próprios estilos e, por conseguinte, suas
próprias estéticas.

$\phantom{linha em branco}$\\ Palavras-chave: Reconhecimento de
padrões. Física estatística. Arte. Pintura. Barroco \& Arte
Moderna. 
%Extração de atributos. Criatividade. Arte e tecnologia.

\end{resumo2}


\afterpage{\blankpage}

\begin{abstract2}
\vspace{-10mm}
VIEIRA, V. \textbf{A quantitative study about the evolution of artistic movements}. 2014. 163p. Dissertação (Mestrado) - Instituto de Física de São Carlos, Universidade de São Paulo, São Carlos, 2014.
\vspace{15mm}

  While many studies were performed for the analysis and classification of
  paintings and other art venues, this study does not stop in the classification
  step. It extends an analysis method already applied to music and philosophy,
  representing the evolution of painting as a time-series where relations like
  \textit{opposition}, \textit{skewness} and \textit{dialectics} were measured
  quantitatively. For that, a corpus of paintings of 12 well-known artists from
  Baroque and Modern art was analyzed. A set of 99 features was extracted and
  the features which most contributed to the classification of painters were
  selected. The projection space obtained from the features provided the basis
  to the analysis of measurements. This quantitative measures underlie revealing
  observations about the evolution of art styles, even when compared with other
  humanity fields already analyzed. While music evolved guided by a
  master-apprentice tradition (high dialectics) and philosophy by opposition,
  painting presents another pattern: constant increasing skewness, low
  opposition between members of the same movement and opposition peaks in the
  transition between movements. A difference between Baroque and Modern
  movements are also observed in the projected ``creative space'': while Baroque
  paintings are presented as an overlapped cluster, the Modern paintings
  presents minor overlapping and are disposed more scattered in the projection
  than baroques. This finds basis in arts history where Baroque painters were
  guided by traditional rules while Modern painters tended to ``break'' these
  rules and develop their own style.

$\phantom{linha em branco}$\\
Keywords: Pattern recognition. Statistical physics. Arts. Painting. Barroque \& Modern Art.

\end{abstract2}
